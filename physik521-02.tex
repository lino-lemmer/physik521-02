\input{header.tex}

\setcounter{section}{1}
\renewcommand\thesection{H\,2.\arabic{section}}
\renewcommand\thesubsection{\thesection.\alph{subsection}}

\title{physik521: Übungsblatt 02}
\author{%
    Lino Lemmer \\ \small{\texttt{s6lilemm@uni-bonn.de}}
    \and
    Martin Ueding \\ \small{\texttt{mu@martin-ueding.de}}
    \and
    Paul Manz \\ \small{\texttt{s6pamanz@uni-bonn.de}}
}

\begin{document}
\maketitle
\section{Entropie des idealen Gases}

\marginpar{M.\,U.}

\subsection{Herleitung der adiabatischen Zustandsänderungen}
\label{ssec:H2.1.a}

\subsubsection{Erste Relation}

Wir beginnen mit der ersten Relation. Wegen $\dif S = 0$ und $\dif N = 0$ folgt:
\begin{align*}
    \dif U &= - p \dif V \\
    \frac f2 N k \dif T &= - p \dif V \\
    \frac f2 \left( \dif p V + p \dif V \right) &= - p \dif V \\
        - \frac 2f \left( \frac f2 + 1 \right) p \dif V &= \dif p V
\end{align*}

Nun können wir aus dem Nichts die Relation folgern:
\begin{align*}
    \left( -1 - \frac 2f + 1 + \frac 2f \right) p \dif V &= 0 \\
    - \frac 2f \left( \frac f2 + 1 \right) p \dif V + \left( 1 + \frac 2f \right) p \dif V &= 0 \\
    \intertext{%
        Jetzt benutzen wir die gerade hergeleitete Relation.
    }
    \dif p V + \left( 1 + \frac 2f \right) p \dif V &= 0 \\
    V^{\frac{f+2}f -1} \left( \dif p V + \frac{f+2}f p \dif V \right) &= 0 \\
    \dif p V^{\frac{f+2}f} + p \frac{f+2}f p V^{\frac{f+2}f} \dif V &= 0 \\
    \intertext{%
        Dies ist nun gerade das totale Differential der
        gesuchten Relation.
    }
    p V^{\frac{f+2}f} &= \text{const}
\end{align*}

\subsubsection{Zweite Relation}

\subsection{Gleichung aus Fundamentalbeziehung}

\subsection{Integration}

\begin{align*}
    \dif s &= \frac 1T \dif U + \frac pT \dif v \\
    \intertext{%
        Wir integrieren auf beiden Seiten.
    }
    \int \dif s &= \int \frac 1T \dif U + \int \frac pT \dif v \\
    \intertext{%
        Nun benutzen wir die Relationen.
    }
    &= k \frac f2 \int \frac NU \dif U + k \int \frac NV \dif v \\
    \intertext{%
        An dieser Stelle substituieren wir $u = U/N$ mit $\dif U = (\dif U N -
        U \dif N)/N^2$, analog $V$.
    }
    &= k \frac f2 \int \frac NU \del{\frac{\dif U}{N} - \frac{U \dif N}{N^2}} + k \int \frac NV \del{\frac{\dif V}{N} - \frac{V \dif N}{N^2}} \\
    &= k \frac f2 \ln\del{\frac U{U_0}} - k \frac f2 \ln\del{\frac N{N_0}} + k \ln\del{\frac V{V_0}} - k \ln\del{\frac N{N_0}} \\
    &= k \del{\frac f2 \ln\del{\frac U{U_0}} + \ln\del{\frac V{V_0}} - \frac{f + 2}f \ln\del{\frac N{N_0}}} \\
    \intertext{%
        Auf der linken Seite gehen wir ähnlich vor: $s = S/N$, $\dif s = (\dif
        S N - S \dif N)/N^2$.
    }
    \int \frac 1N \dif S - \int \frac S{N^2} \dif N &= \\
    \eval{\frac SN} + \eval{\frac SN} &= \\
    \frac SN - \frac{S_0}{N_0} &= \\
    S &= S_0 \frac{N}{N_0} + k \del{\frac f2 \ln\del{\frac U{U_0}} + \ln\del{\frac V{V_0}} - \frac{f + 2}f \ln\del{\frac N{N_0}}}
\end{align*}

\section{Otto-Zyklus}

\subsection{Adiabatische Kompression}

Adiabatisch heißt hier $\dif Q = 0$. Die Änderung der inneren Energie $\Delta U$ geht also vollständig in mechanische Arbeit über:

\begin{align}
    \Delta W_{\text{A}\to\text{B}} &= -\Delta U_{\text{A}\to\text{B}} = \frac f2 Nk_\text{B}\del{T_\text{A}-T_\text{B}} \notag
    \intertext{Zudem gilt die in \ref{ssec:H2.1.a} hergeleitete Beziehung}
    VT^{\frac f2} &= \text{const} \notag
    \intertext{sodass gilt}
    \frac{V_\text{A}}{V_\text{B}} &= \del{ \frac{T_\text{A}}{T_\text{B}}}^{-\frac f2} \label{eq:VA/VB}
\end{align}

\subsection{Isochore Erwärmung}

Isochor bedeutet $\Delta V = 0$. Es wird also keine mechanische Arbeit verrichtet.

\[
    \Delta Q_{\text{B}\to\text{C}} = \Delta U = \frac f2 Nk_\text{B}\del{T_\text{C}-T_\text{B}}
\]

\subsection{Adiabatische Expansion}

\begin{align}
    \Delta W_{\text{C}\to\text{D}} &= -\Delta U_{\text{C}\to\text{D}}  = \frac f2 Nk_\text{B}\del{T_\text{C}-T_\text{D}} \notag\\
    \frac{V_\text{C}}{V_\text{D}} &= \del{ \frac{T_\text{C}}{T_\text{D}}}^{-\frac f2} \label{eq:VC/VD}
\end{align}

\subsection{Isochore Abkühlung}

\[
    \Delta Q_{\text{D}\to\text{A}}  = \Delta U_{\text{D}\to\text{A}}   = \frac f2 Nk_\text{B}\del{T_\text{A}-T_\text{D}}
\]

\subsection{Gesamter Prozess}

Die Gesamte am System verrichtete Arbeit ist

\begin{align}
    W &= \Delta W_{\text{A}\to\text{B}} + \Delta W_{\text{C}\to\text{D}} \notag\\
      &= \frac f2 Nk_\text{B}\del{T_\text{A} + T_\text{C} - T_\text{B} - T_\text{D}} \notag
    \intertext{Die eingebrachte Wärmeenergie ist}
    Q_\text{in} &= \Delta Q_{\text{B}\to\text{C}} \notag\\
                &= \frac f2 Nk_\text{B}\del{T_\text{C}-T_\text{B}} \notag
    \intertext{Damit ist der Wirkungsgrad}
    \eta_\text{Otto} &= \frac{W}{Q_\text{in}} \notag\\
    &= \frac {T_\text{B} + T_\text{D} - T_\text{A} - T_\text{C}}
    {T_\text{C}-T_\text{B}} \label{eq:eta_Otto}
    \intertext{Da im Otto-Zyklus $V_\text{A}=V_\text{D}$ und $V_\text{B} = V_\text{C}$ gilt folgt aus den Gleichungen~\eqref{eq:VA/VB}~und~\eqref{eq:VC/VD}}
    \frac{T_\text{A}}{T_\text{B}} &= \frac{T_\text{D}}{T_\text{C}} \notag
    \intertext{Sei nun $T_\text{D} = \alpha T_\text{A}$ Dann ist auch $T_\text{C} = \alpha T_\text{B}$. Damit ergibt sich aus Gleichung \eqref{eq:eta_Otto}:}
    &= \frac{\del{1-\alpha}T_\text{B} - \del{1-\alpha}T_\text{A}}{\del{1-\alpha}T_\text{B}} \notag\\
    &= 1-\frac{T_\text{A}}{T_\text{B}}\notag
    \intertext{Mit Gleichung \eqref{eq:VA/VB} erhält man}
    &= 1 - \del{ \frac{V_\text{B}}{V_\text{A}} }^{\frac 2f} \notag
    \intertext{Mit $\frac 2f = \frac{f+2}f-1 = \kappa-1$ ergibt sich die gesuchte Relation}
    &= 1 - \del{ \frac{V_\text{B}}{V_\text{A}} }^{\kappa-1} \notag
\end{align}


\section{Inverser Carnot-Prozess als Wärmepumpe}

\subsection{}

\subsection{}

\end{document}
