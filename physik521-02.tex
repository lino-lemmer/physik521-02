% Für Seitenformatierung

\documentclass[DIV=15]{scrartcl}

% Zeilenumbrüche

\parindent 0pt
\parskip 6pt

% Für deutsche Buchstaben und Synthax

\usepackage[ngerman]{babel}

% Für Auflistung mit speziellen Aufzählungszeichen

\usepackage{paralist}

% zB für \del, \dif und andere Mathebefehle

\usepackage{amsmath}
\usepackage{commath}
\usepackage{amssymb}

% für nicht kursive griechische Buchstaben

\usepackage{txfonts}

% Für \SIunit[]{} und \num in deutschem Stil

\usepackage[output-decimal-marker={,}]{siunitx}
\usepackage[utf8]{inputenc}

% Für \sfrac{}{}, also inline-frac

\usepackage{xfrac}

% Für Einbinden von pdf-Grafiken

\usepackage{graphicx}

% Umfließen von Bildern

\usepackage{floatflt}

% Für Links nach außen und innerhalb des Dokumentes

\usepackage{hyperref}

% Für weitere Farben

\usepackage{color}

% Für Streichen von z.B. $\rightarrow$

\usepackage{centernot}

% Für Befehl \cancel{}

\usepackage{cancel}

% Für Layout von Links

\hypersetup{
	citecolor=black,
	colorlinks=true,
	linkcolor=black,
	urlcolor=blue,
}

% Verschiedene Mathematik-Hilfen

\newcommand \e[1]{\cdot10^{#1}}
\newcommand\p{\partial}

\newcommand\half{\frac 12}
\newcommand\shalf{\sfrac12}

\newcommand\skp[2]{\left\langle#1,#2\right\rangle}
\newcommand\mw[1]{\left\langle#1\right\rangle}
\renewcommand \exp[1]{\mathrm e^{#1}}

% Nabla und Kombinationen von Nabla

\renewcommand\div[1]{\skp{\nabla}{#1}}
\newcommand\rot{\nabla\times}
\newcommand\grad[1]{\nabla#1}
\newcommand\laplace{\triangle}
\newcommand\dalambert{\mathop{{}\Box}\nolimits}

%Für komplexe Zahlen

\renewcommand \i{\mathrm i}
\renewcommand{\Im}{\mathop{{}\mathrm{Im}}\nolimits}
\renewcommand{\Re}{\mathop{{}\mathrm{Re}}\nolimits}

%Für Bra-Ket-Notation

\newcommand\bra[1]{\left\langle#1\right|}
\newcommand\ket[1]{\left|#1\right\rangle}
\newcommand\braket[2]{\left\langle#1\left.\vphantom{#1 #2}\right|#2\right\rangle}
\newcommand\braopket[3]{\left\langle#1\left.\vphantom{#1 #2 #3}\right|#2\left.\vphantom{#1 #2 #3}\right|#3\right\rangle}


\setcounter{section}{1}
\renewcommand\thesection{H\,2.\arabic{section}}
\renewcommand\thesubsection{\thesection.\alph{subsection}}

\title{physik521: Übungsblatt 02}
\author{%
    Lino Lemmer \\ \small{\texttt{s6lilemm@uni-bonn.de}}
    \and
    Martin Ueding \\ \small{\texttt{mu@martin-ueding.de}}
}

\begin{document}
\maketitle
\section{Entropie des idealen Gases}

\marginpar{M.\,U.}

\subsection{Herleitung der adiabatischen Zustandsänderungen}

Wir beginnen mit der ersten Relation. Wegen $\dif S = 0$ und $\dif N = 0$ folgt:
\begin{align*}
    \dif U &= - p \dif V \\
    \frac f2 N k \dif T &= - p \dif V \\
    \frac f2 \left( \dif p V + p \dif V \right) &= - p \dif V \\
        - \frac 2f \left( \frac f2 + 1 \right) p \dif V &= \dif p V
\end{align*}

Nun können wir aus dem Nichts die Relation folgern:
\begin{align*}
    \left( -1 - \frac 2f + 1 + \frac 2f \right) p \dif V &= 0 \\
    - \frac 2f \left( \frac f2 + 1 \right) p \dif V + \left( 1 + \frac 2f \right) p \dif V &= 0 \\
    \intertext{%
        Jetzt benutzen wir die gerade hergeleitete Relation.
    }
    \dif p V + \left( 1 + \frac 2f \right) p \dif V &= 0 \\
    V^{\frac{f+2}f -1} \left( \dif p V + \frac{f+2}f p \dif V \right) &= 0 \\
    \dif p V^{\frac{f+2}f} + p \frac{f+2}f p V^{\frac{f+2}f} \dif V &= 0 \\
    \intertext{%
        Dies ist nun gerade das totale Differential der
        gesuchten Relation.
    }
    p V^{\frac{f+2}f} &= \text{const}
\end{align*}

\subsection{Gleichung aus Fundamentalbeziehung}

\subsection{Integration}

\begin{align*}
    \dif s &= \frac 1T \dif U + \frac pT \dif v \\
    \intertext{%
        Wir integrieren auf beiden Seiten.
    }
    \int \dif s &= \int \frac 1T \dif U + \int \frac pT \dif v \\
    \intertext{%
        Nun benutzen wir die Relationen.
    }
    &= k \frac f2 \int \frac NU \dif U + k \int \frac NV \dif v \\
    \intertext{%
        An dieser Stelle substituieren wir $u = U/N$ mit $\dif U = (\dif U N -
        U \dif N)/N^2$, analog $V$.
    }
    &= k \frac f2 \int \frac NU \del{\frac{\dif U}{N} - \frac{U \dif N}{N^2}} + k \int \frac NV \del{\frac{\dif V}{N} - \frac{V \dif N}{N^2}} \\
    &= k \frac f2 \ln\del{\frac U{U_0}} - k \frac f2 \ln\del{\frac N{N_0}} + k \ln\del{\frac V{V_0}} - k \ln\del{\frac N{N_0}} \\
    &= k \del{\frac f2 \ln\del{\frac U{U_0}} + \ln\del{\frac V{V_0}} - \frac{f + 2}f \ln\del{\frac N{N_0}}} \\
    \intertext{%
        Auf der linken Seite gehen wir ähnlich vor: $s = S/N$, $\dif s = (\dif
        S N - S \dif N)/N^2$.
    }
    \int \frac 1N \dif S - \int \frac S{N^2} \dif N &= \\
    \eval{\frac SN} + \eval{\frac SN} &= \\
    \frac SN - \frac{S_0}{N_0} &= \\
    S &= S_0 \frac{N}{N_0} + k \del{\frac f2 \ln\del{\frac U{U_0}} + \ln\del{\frac V{V_0}} - \frac{f + 2}f \ln\del{\frac N{N_0}}}
\end{align*}

\section{Otto-Zyklus}

\section{Inverser Carnot-Prozess als Wärmepumpe}

\subsection{}

\subsection{}

\end{document}
